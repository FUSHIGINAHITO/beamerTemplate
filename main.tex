\documentclass[10pt]{beamer}

\usepackage{fontawesome}

% \makeatletter
% \g@addto@macro\normalsize{%
%     \setlength\belowdisplayskip{-0pt}
% }

\usepackage[english]{babel}
\usepackage{mathrsfs}
\usepackage{mathtools}
\usepackage{amsmath}
\usepackage{ragged2e} \justifying\let\raggedright\justifying 

\usefonttheme{structuresmallcapsserif}
\setbeamerfont{title}{family=\fontfamily{bch}\selectfont}
\setbeamerfont{frametitle}{family=\fontfamily{bch}\selectfont}
\setbeamerfont{framesubtitle}{family=\fontfamily{bch}\selectfont}
\setbeamerfont{framesubsubtitle}{family=\fontfamily{bch}\selectfont}
\setbeamerfont{footline}{family=\fontfamily{bch}\selectfont}
\setbeamerfont{subsection in toc}{family=\fontfamily{bch}\selectfont}
\setbeamerfont{section in toc}{family=\fontfamily{bch}\selectfont}
\usefonttheme{professionalfonts}
\renewcommand*\sfdefault{cmbr}

% 去掉下面没用的导航条
\setbeamertemplate{navigation symbols}{}
% 设置页脚格式
\makeatother
\setbeamertemplate{footline}
{
  \leavevmode%
  \hbox{%
  \begin{beamercolorbox}[wd=\paperwidth,ht=2.25ex,dp=1ex,center]{title in head/foot}%
    \usebeamerfont{title in head/foot}\insertshorttitle
    \hfill
    \insertframenumber{} / \inserttotalframenumber\hspace*{0ex}
  \end{beamercolorbox}}

  \vskip0pt%
}
\makeatletter

\renewcommand{\emph}[1]{\textbf{\color{color5}#1}}

% 定义颜色
%\definecolor{alizarin}{rgb}{0.82, 0.1, 0.26} % 红色
%\definecolor{DarkFern}{HTML}{407428} % 绿色
%\colorlet{main}{DarkFern!100!white} % 第一种设置方法
%\colorlet{main}{red!70!black} % 第二种设置方法
\definecolor{bistre}{rgb}{0.24, 0.17, 0.12}

\definecolor{color1}{RGB}{93, 113, 170}
\definecolor{color2}{RGB}{135, 197, 237}
\definecolor{color3}{RGB}{241, 182, 194}
\definecolor{color4}{RGB}{197, 224, 169}
\definecolor{color5}{RGB}{222, 32, 120}
\definecolor{color6}{RGB}{233, 203, 153}

\colorlet{main}{white}
\colorlet{text}{bistre!100!white}

% 不同元素指定不同颜色,fg是本身颜色,bg是背景颜色,!num!改变数值提供渐变色
\setbeamercolor{title}{fg=main}
\setbeamercolor{frametitle}{fg=main}
\setbeamercolor{section in toc}{fg=text}
\setbeamercolor{normal text}{fg=text}
\setbeamercolor{qed symbol}{fg=color4} % 证明结束后的框颜色
%\setbeamercolor{math text}{fg=black} % 奇异代码,导致tcolorbox中的equation多一个空行
% 设置页脚对应位置颜色
\setbeamercolor{title in head/foot}{fg=main, bg=color1}
\setbeamercolor{structure}{fg=main, bg=color1} % 设置sidebar颜色

% 左右页间距的排版
\def\swidth{0cm}
\setbeamersize{sidebar width right=\swidth}
\setbeamersize{sidebar width left=\swidth}
\setbeamerfont{title in sidebar}{size=\scriptsize}
\setbeamerfont{section in sidebar}{size=\tiny}

% tcolorbox相关
\usepackage{etoolbox,amssymb}
\usepackage[many]{tcolorbox}


% 不带证毕符号的proof
\newtheorem{myproofcut}{Proof}[section]
\makeatletter
\renewtcolorbox[use counter=myproofcut,number within=section]{myproofcut}[1][]{
  title={Proof\ #1},
  colback=color4!10!white,
  colframe=color4,
  fonttitle=\fontfamily{bch}\selectfont\bfseries,
}
\makeatother

\usepackage{amssymb}% defines \blacksquare
\renewcommand{\qedsymbol}{{$\color{color4}\blacksquare$}}
\AtBeginEnvironment{myproof}{\pushQED{\qed}}
\AtEndEnvironment{myproof}{\popQED}
\newtheorem{myproof}{Proof}[section]
\makeatletter
\renewtcolorbox[use counter=myproof,number within=section]{myproof}[1][]{
  title={Proof\ #1},
  colback=color4!10!white,
  colframe=color4,
  fonttitle=\fontfamily{bch}\selectfont\bfseries,
}
\makeatother

\newtheorem{mysol}{Proof}[section]
\makeatletter
\renewtcolorbox[use counter=mysol,number within=section]{mysol}[1][]{
  title={Solution\ #1},
  colback=color4!10!white,
  colframe=color4,
  fonttitle=\fontfamily{bch}\selectfont\bfseries,
}
\makeatother

\newtheorem{mydef}{Definition}[section]
\makeatletter
\renewtcolorbox[use counter=mydef,number within=section]{mydef}[1][]{
  title={Definition \themydef\ #1},
  colback=color3!10!white,
  colframe=color3,
  fonttitle=\fontfamily{bch}\selectfont\bfseries,
}
\makeatother

\newtheorem{mythm}{Theorem}[section]
\makeatletter
\renewtcolorbox[use counter=mythm,number within=section]{mythm}[1][]{
  title={Theorem \themythm\ #1},
  colback=color2!10!white,
  colframe=color2,
  fonttitle=\fontfamily{bch}\selectfont\bfseries,
}
\makeatother

\newtheorem{mylem}{Theorem}[section]
\makeatletter
\renewtcolorbox[use counter=mylem,number within=section]{mylem}[1][]{
  title={Lemma \themylem\ #1},
  colback=color2!10!white,
  colframe=color2,
  fonttitle=\fontfamily{bch}\selectfont\bfseries,
}
\makeatother

\newtheorem{mycol}{Theorem}[section]
\makeatletter
\renewtcolorbox[use counter=mycol,number within=section]{mycol}[1][]{
  title={Corollary \themycol\ #1},
  colback=color2!10!white,
  colframe=color2,
  fonttitle=\fontfamily{bch}\selectfont\bfseries,
}
\makeatother

\newtheorem{myfact}{Theorem}[section]
\makeatletter
\renewtcolorbox[use counter=myfact,number within=section]{myfact}[1][]{
  title={Fact \themyfact\ #1},
  colback=color2!10!white,
  colframe=color2,
  fonttitle=\fontfamily{bch}\selectfont\bfseries,
}
\makeatother

\newtheorem{mypro}{Theorem}[section]
\makeatletter
\renewtcolorbox[use counter=mypro,number within=section]{mypro}[1][]{
  title={Proposition \themypro\ #1},
  colback=color2!10!white,
  colframe=color2,
  fonttitle=\fontfamily{bch}\selectfont\bfseries,
}
\makeatother

\newtheorem{mycon}{Theorem}[section]
\makeatletter
\renewtcolorbox[use counter=mycon,number within=section]{mycon}[1][]{
  title={Conjecture \themycon\ #1},
  colback=color2!10!white,
  colframe=color2,
  fonttitle=\fontfamily{bch}\selectfont\bfseries,
}
\makeatother

\newtheorem{myclaim}{Theorem}[section]
\makeatletter
\renewtcolorbox[use counter=myclaim,number within=section]{myclaim}[1][]{
  title={Claim \themyclaim\ #1},
  colback=color2!10!white,
  colframe=color2,
  fonttitle=\fontfamily{bch}\selectfont\bfseries,
}
\makeatother

\newtheorem{myexa}{Theorem}[section]
\makeatletter
\renewtcolorbox[use counter=myexa,number within=section]{myexa}[1][]{
  title={Example \themyexa\ #1},
  colback=color6!10!white,
  colframe=color6,
  fonttitle=\fontfamily{bch}\selectfont\bfseries,
}
\makeatother

\newtheorem{myrem}{Theorem}[section]
\makeatletter
\renewtcolorbox[use counter=myrem,number within=section]{myrem}[1][]{
  title={Remark \themyrem\ #1},
  colback=color6!10!white,
  colframe=color6,
  fonttitle=\fontfamily{bch}\selectfont\bfseries,
}
\makeatother

% 去掉章节标号
\defbeamertemplate{section page}{mine}[1][]{%
  \begin{centering}
    {\usebeamerfont{section name}\usebeamercolor[fg]{section name}#1}
    \vskip1em\par
    \begin{beamercolorbox}[sep=12pt,center]{part title}
      \usebeamerfont{section title}\insertsection\par
    \end{beamercolorbox}
  \end{centering}
}

\defbeamertemplate{subsection page}{mine}[1][]{%
  \begin{centering}
    {\usebeamerfont{subsection name}\usebeamercolor[fg]{subsection name}#1}
    \vskip1em\par
    \begin{beamercolorbox}[sep=8pt,center,#1]{part title}
      \usebeamerfont{subsection title}\insertsubsection\par
    \end{beamercolorbox}
  \end{centering}
}

% 修改item样式
\setbeamertemplate{itemize item}{\color{color5}\faSunO}
\setbeamertemplate{itemize subitem}{\color{color5}\faMoonO}
\setbeamertemplate{itemize subsubitem}{\color{color5}\faStarO}
\setbeamerfont{enumerate item}{family=\fontfamily{bch}\selectfont}
\setbeamerfont{enumerate subitem}{family=\fontfamily{bch}\selectfont}
\setbeamerfont{enumerate subsubitem}{family=\fontfamily{bch}\selectfont}
\setbeamercolor{enumerate item}{fg=color5}
\setbeamercolor{enumerate subitem}{fg=color5}
\setbeamercolor{enumerate subsubitem}{fg=color5}

%-------------------正文-------------------------%

\author{Your Name}
\title{A very looooooooooong  looooooooooong  looooooooooong Presentation Title}
\date{\today}

\begin{document}
\setbeamertemplate{section page}[mine]
\setbeamertemplate{subsection page}[mine]

\frame[plain]{\titlepage}

\begin{frame}
\frametitle{Outline}
\tableofcontents
\end{frame}

\section{Page Title}

\begin{frame}
\sectionpage
\end{frame}

\begin{frame}
\frametitle{Page Title}

TeX - LaTeX Stack Exchange is a question and answer site for users of TeX, LaTeX, ConTeXt, and related typesetting systems.

\vspace{0.4cm}

unordered list below

\begin{itemize}
\item The first item
\item The second item
\item The third item
\item The fourth item
\begin{itemize}
\item The first item
\item The second item
\item The third item
\item The fourth item
\begin{itemize}
\item The first item
\item The second item
\item The third item
\item The fourth item
\end{itemize}
\end{itemize}
\end{itemize}

\end{frame}

\section{Display Theorem}

\frame{\frametitle{Outline}\tableofcontents[currentsection]}

\subsection{First subsection}

\subsection{Second subsection}

\begin{frame}{Frame Title}
    \subsectionpage
\end{frame}

\begin{frame}
  \frametitle{Display Theorem}
  \begin{mythm}
    This is a text in second frame. $1 + 2 = 3$
  \end{mythm}
  \begin{myproof}
    This is a text in second frame. $1 + 1 + 1 = 3$
    \begin{displaymath}
    1 + 1 = 2\qedhere
    \end{displaymath}
  \end{myproof}
  \begin{myproofcut}
    This is a text in second frame. 
    \begin{displaymath}
    1 + 1 = 2
    \end{displaymath}%
  \end{myproofcut}
\end{frame}

\section{Sample frame title}

\frame{\frametitle{Outline}\tableofcontents[currentsection]}

\begin{frame}
\frametitle{Sample frame title}
This is a text in second frame.
For the sake of showing an example.

\begin{enumerate}
 \itemsep0em
 \item Text visible on slide 1
 \item Text visible on slide 2
 \item Text visible on slide 3
 \begin{enumerate}
 \itemsep0em
 \item Text visible on slide 1
 \item Text visible on slide 2
 \item Text visible on slide 3
 \begin{enumerate}
 \itemsep0em
 \item Text visible on slide 1
 \item Text visible on slide 2
 \item Text visible on slide 3
\end{enumerate}
\end{enumerate}
\end{enumerate}
\end{frame}

\begin{frame}
  \begin{mydef}[(Gaussian Elimination)]
    $$
 \frac{1}{\displaystyle 1+
   \frac{1}{\displaystyle 2+
   \frac{1}{\displaystyle 3+x}}} +
 \frac{1}{1+\frac{1}{2+\frac{1}{3+x}}}
$$
$$\int_0^\infty e^{-x^2} dx=\frac{\sqrt{\pi}}{2}$$
\begin{equation} x=y+3 \label{eq:xdef}
\end{equation}
In equation (\ref{eq:xdef}) we saw $\dots$
  \end{mydef}

  \begin{myrem}
        This is a text in second frame.
        For the sake of showing an example.
        $x=y+3$
  \end{myrem}
  
\end{frame}

\begin{frame}{Frame Title}
    \begin{myexa}
        This is a text in second frame.
        For the sake of showing an example.
        $x=y+3$
    \end{myexa}

    \begin{mycol}
        This is a text in second frame.
        For the sake of showing an example.
        $x=y+3$
    \end{mycol}

    \begin{mylem}
        This is a text in second frame.
        For the sake of showing an example.
        $x=y+3$
    \end{mylem}
\end{frame}

\begin{frame}{Frame Title}
    \begin{myfact}
        This is a text in second frame.
        For the sake of showing an example.
        $x=y+3$
    \end{myfact}

    \begin{mycon}
        This is a text in second frame.
        For the sake of showing an example.
        $x=y+3$
    \end{mycon}

    \begin{mypro}
        This is a text in second frame.
        For the sake of showing an example.
        $x=y+3$
    \end{mypro}
\end{frame}

\begin{frame}{Frame Title}
    \begin{myclaim}
        This is a text in second frame.
        For the sake of showing an example.
        $$x=y+3$$
    \end{myclaim}

    \begin{mysol}
        This is a \emph{text} in second frame.
        For the sake of showing an example.
        $$x=y+3$$
    \end{mysol}
\end{frame}


\end{document}
